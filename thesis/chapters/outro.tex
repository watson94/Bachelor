\startconclusionpage

Гонки в программа являются причинами убытков. Поэтому обнаружение гонок является актуальной задачей. Работы в области динамического обнаружения в настоящее время ведутся активно.  Главной проблемой динамического подхода является ущерб, наносимый выполнению анализируемой программы. 
В данной работе был рассмотрен вопрос оптимизации динамических детекторов, путем проведения предварительного  статического анализа.

В рамках работы был разработан алгоритм для поиска корректно-синхронизированных полей. Данный алгоритм позволяет обнаруживать поля, при обращении к которым не может возникнуть гонки. Данный анализ позволяет сократить время выполнения и объем потребления памяти динамического детектора.

Полученный алгоритм позволяет отслеживать различные операции синхронизации, используемые для защиты обращений к разделяемым переменным. Алгоритм основан на обходе графов потока исполнения методов. 

Также была разработана программа, реализующая разработанный алгоритм. 
Корректность работы программы была протестирована большим набором тестов. Получены результаты запуска программы на типичных приложениях, которые подтвердили предположения о возможностях данного подхода.

Одной из целей данной работы являлась оптимизация динамического поиска гонок. Полученная программа предоставляет интерфейс для интеграции с динамическими детекторами. В рамках данной работы она была интегрирована с существующим динамическим детектором гонок для java программ. Статистика, полученная динамическим детектором, подтвердила 
прирост его производительности.  

Таким образом, главными практческими результатами являются разработання программа и полученная оптимизация динамического детектора. Гланым теоретическим результатом является предложенный алгоритм для поиска корректно-синхронизированных полей. 

\FloatBarrier
