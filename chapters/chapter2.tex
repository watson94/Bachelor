%-*-coding: utf-8-*-
\chapter{Решение массовой задачи о ближайшем отрезке с использованием квадродерева}
\section{Квадродерево}
Квадродерево -- поисковая структура данных, которая хранит в себе
подразбиение плоскости и позволяет быстро производить локализацию точек-
запросов. Узел (ячейка) квадродерева представляет собой прямоугольник, для
которого определена некоторая мера его насыщенности $p(C)$. Если узел
насыщен ($p(C) > T$, где $T$ -- предельное насыщение), то происходит его
разбиение на четыре одинаковых дочерних узла (делением пополам по
вертикали и по горизонтали). Таким образом, в каждый момент времени у узла
или нет детей или их четыре. Разбиение происходит до тех пор, пока все узлы
не перестанут быть насыщенными или не будет достигнута максимальная
глубина подразбиения. Ограничение глубины подразбиения играет важную
роль в виду того, что не всегда получается сделать узел ненасыщенным за
конечное (или разумное) количество разбиений, при описании применения
квадродерева в предложенном алгоритме, будет дано более точное обоснование
необходимости ограничения.

Квадродеревья и их модификации очень часто применяют для решения
задач примерного поиска ближайшего соседа (Approximate Nearest Neighbor
Search) для точек \cite{FANN} (рис. \ref{ann}). В качестве меры насыщения в этой задаче часто
выбирают количество точек, среди которых производится поиск, попавших в
ячейку. В насыщенность обычно ограничивают одной точкой в одной ячейке.
Максимальная глубина древа для $n$ точек может составлять $n$, в результате чего
время локализации может составлять $O(n)$. Для борьбы с этим была разработана
структура данных Skip-Quadtree \cite{SQT}, которая позволяет производить локализацию за $O(\log n)$.

\drawfigure{ann}{Квадродерево}{ann}

\section{Нижняя огибающая}
Неформально нижняя огибающая (lower envelope) множества объектов на плоскости –
множество точек этих объектов, видимые наблюдателем, расположенным в
точке $(0, -\infty)$. Формально же это граф, представляющий из себя поточечный
минимум кусочно-заданных функций \cite{LENV} (рис. \ref{lenv}).
Также наряду с нижней огибающей часто рассматривается минимизационная диаграмма
(minimization diagram), которая представляет собой проекцию нижней огибающей на
горизонтальную ось (рис. \ref{mdiag}). 
\drawfigure{lenv}{Нижняя огибающая}{lenv}
\drawfigure{mdiag}{Минимизационная диаграмма}{mdiag}

\section{Алгоритм}
\subsection{Идея алгоритма}
Основной идеей всех алгоритмов поиска ближайших сайтов (sites),
основанных на подразбиении пространства, является растеризация (с явным
построением или без него) диаграммы Вороного в этом подразбиении. После
этого в ячейках подразбиения оказывается информация, обо всех ближайших
сайтов для всех точек этой ячейки. Поиск ближайшего сайта происходит
путем локализации в этом подразбиении и последующим перебором всех
сайтов, ближайших к найденной ячейке.

В виду нетривиальности задачи поиска всех сайтов ближайших к ячейке,
во многих алгоритмах переходят к примерному решению задачи поиска
ближайшего отрезка \cite{NGRID}, производя поиск сайтов, ближайших к каким-то
точкам ячейки. Точки обычно выбираются таким образом, чтобы обеспечить
заданную точность, но в некоторых случаях даже не идет речи о точности \cite{AVOR}.
Для некоторых случаев погрешность допустима, но робастность (robustness)
является важной характеристикой алгоритмов вычислительной геометрии \cite{ROBUS}.
Предложенный алгоритм позволяет произвести точный поиск ближайших
отрезков для ячеек, при условии, что можно явно (хотя бы кусочно) задать
расстояние от границ ячеек до отрезков в виде полинома.

\subsection{Работа алгоритма}

Для отрезков строится ограничивающий прямоугольник (bounding box), этот прямоугольник будет первым
уровнем квадродерева. В качестве меры насыщенности узла берется количество
ближайших отрезков к данной ячейке. Для первого узла ближайшими будут все
отрезки, так как они все лежат внутри. Далее происходит рекурсивное
подразбиение узлов. Ближайшими к дочернему узлу будут отрезки ближайшие
к его родителю, так как дочерний узел геометрически лежит внутри
родительского. Необходимо произвести фильтрацию лишних отрезков.

{\prop\label{cl_segs}
Ближайшие отрезки для точек ячейки – это отрезки ближайшие к ее границе и отрезки пересекающие ячейку}
\proof Обозначим: $S$ -- множество отрезков ближайших к ячейке, $S_b$ -- ближайших к границе, $S_i$ --
пересекающих ячейку.
\begin{itemize}
\item $S_b \cup S_i \subset S$ \\
$S_b \subset S$ -- очевидно, так как граница ячейки -- это ее подмножество.\\
Для любой точки на пересечении отрезка и ячейки этот отрезок будет
ближайшим, значит $S_i \subset S$
\item $S_b \cup S_i \supset S$ \\
Предположим, что это не так.\\Пусть $s$ -- отрезок, не пересекающий
ячейку, и он не является ближайшим ни к одной точке на границе. Пусть
он ближайший для точки $P$ ячейки, а $Q$ -- точка $s$, ближайшая к $P$.
Построим отрезок $PQ$, так точка $P$ вне ячейки, а точка $Q$ внутри, то $PQ$
пересечет границу ячейки, допустим в точке $F$. Рассмотрим отрезок $s'$,
ближайший к $F$. Пусть точка $E$ -- ближайшая точка на нем к $F$.
Так как $s'$ ближайший к $F$, то $|FE| < |FQ|$, по неравенству треугольника $|PF|
+ |FE| < |PE|$. Подставив первое неравенство во второе, мы получим, что
отрезок $s'$ ближе к $P$ чем $s$ (рис. \ref{contrex}). Противоречие.
\end{itemize}
\drawfigure{contrex}{Противоречие}{contrex}

Это простое утверждение показывает, что для фильтрации нам
необходимо взять из отрезков только те, которые являются ближайшими для
границы ячейки, и те, которые ее пересекают.
Для поиска отрезков ближайших к границе ячейки для каждой стороны
прямоугольника строится нижняя огибающая функций кратчайшего расстояния от
стороны до отрезков, которые фильтруются (рис. \ref{le_dist}).

Функция кратчайшего расстояния до отрезка состоит из трех частей: двух функции расстояния от
стороны до концов отрезка, и функции расстояния от стороны до прямой,
содержащей отрезок этот, заданной на ограниченном промежутке. В результате
из нижней огибающей можно выделить информацию об отрезках ближайших к
сторонам ячейки. Также эта фильтрация оставляет отрезки, пересекающие
границу. Значит, к полученным отрезкам остается только добавить отрезки
лежащие внутри ячейки. Для проверки этого условия достаточно проверить
принадлежность одной из точек отрезка ячейке.

\drawfigurex{le_dist}{Нижняя огибающа функций кратчайших расстояний}{le_dist}{width=6cm}

Подразбиение будет происходить до тех пор, пока все ячейки не
перестанут быть насыщенными, или пока не будет достигнута максимальная
глубина подразбиения. В данной задаче очень важно ограничить глубину
подразбиения, так как в вырожденных случаях (degenerate cases) некоторые
ячейки поразбить не получится. Вырожденным случаем для диаграммы
Вороного является наличие четырех и более сайтов равноудаленных от одной
точки. В таком случае в этой точке получается вершина диаграммы Вороного,
граничащая с ячейками соответствующих сайтов. В результате наличия
большого числа сайтов расположенных таким образом (пусть их $n$), ячейка
квадродерева, содержащая эту точку, будет ближайшей как минимум к $n$
сайтам. Разбив такую ячейку мы все равно получим одну ячейку содержащую
эту вершину диаграммы Вороного. Поэтому имеет смысл ограничивать
глубину разбиения.
Итак, в результате получается квадродерево, в листьях которого лежит
информация о ближайших к ним отрезках. Поиск ближайшего отрезка по такой
структуре данных осуществляется в два этапа. Сначала происходит
локализация точки-запроса в квадродереве. Затем перебираются все отрезки,
ближайшие к найденной ячейке, и среди них выбирается ближайший.

\subsection{Анализ полученных результатов}
Если произвести грубую оценку времени построения квадродерева для
этой задачи, то получается, что оно ограничено только максимальной глубиной
подразбиения. Это так, но на практике построение происходит достаточно
быстро, в виду того, что сильное подразбиение испытывают в основном
области, содержащие вершины SVD. Максимально возможное число ячеек
будет $4^d$, где $d$ -- максимальная глубина подразбиения.

Скорость поиска ближайшего отрезка складывается из скорости
локализации и скорости поиска ближайшего отрезка среди ближайших к
ячейке. Тогда как первая величина ограничена сверху $d$, вторая ограничивается
только количеством отрезков (достаточно вспомнить вырожденный случай).
Ввиду того что $d$ обычно не очень велико и локализация в дереве -- не
ресурсоемкая операция, очень показательной величиной оказывается среднее
число отрезков перебираемых при запросе. Эта величина является суммой по
всем листовым ячейкам дерева количества ближайших отрезков, помноженных
на долю площади, покрытой этой ячейкой. На практике такая структура
оказывается очень эффективной, она позволяет давать точные ответы на
запросы о ближайших отрезках в среднем за $O(1)$, а среднее число
перебираемых отрезков держится в промежутке $0,5k ÷ k$, где $k$ -- предельная
насыщенность ячейки.

В следующей главе будет приведено сравнение
квадродерева с n-grid \cite{NGRID} и SVD \cite{CGAL}.
Максимальное количество занимаемой памяти можно грубо оценить, как
O($4^{d}n$), что на практике не наблюдается.
