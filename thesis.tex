% -*-coding: utf-8-*-
% This is an AMS-LaTeX v. 1.2 File.

\documentclass{report}

%\usepackage{pscyr}
%\renewcommand{\rmdefault}{fjn}
%\renewcommand{\ttdefault}{fcr}

%\usepackage{showkeys}
\usepackage[T2A]{fontenc}
\usepackage[utf8x]{inputenc}
\usepackage[english,russian]{babel}
\usepackage{expdlist}
\usepackage[dvips]{graphicx}
\usepackage{amsmath}
\usepackage{amssymb}
\usepackage{amsthm}
\usepackage{amsfonts}
\usepackage{amsxtra} 
\usepackage{sty/dbl12}
\usepackage{srcltx}
\usepackage{epsfig}
\usepackage{verbatim}
\usepackage{sty/rac}
%\usepackage[russian]{sty/ralg}
\usepackage{listings}


%%%%%%%%%%%%%%%%%%%%%%%%%%%%%%%%%%%%%%%%%%%%%%%%%%%%%%%%%%%%%%%%%%%%%%%%%%%%%%

% Redefine margins and other page formatting

\setlength{\oddsidemargin}{0.5in}

% Various theorem environments. All of the following have the same numbering
% system as theorem.

\theoremstyle{plain}
\newtheorem{theorem}{Теорема}
\newtheorem{prop}[theorem]{Утверждение}
\newtheorem{corollary}[theorem]{Следствие}
\newtheorem{lemma}[theorem]{Лемма}
\newtheorem{question}[theorem]{Вопрос}
\newtheorem{conjecture}[theorem]{Гипотеза}
\newtheorem{assumption}[theorem]{Предположение}

\theoremstyle{definition}
\newtheorem{definition}[theorem]{Определение}
\newtheorem{notation}[theorem]{Обозначение}
\newtheorem{condition}[theorem]{Условие}
\newtheorem{example}[theorem]{Пример}
\newtheorem{algorithm}[theorem]{Алгоритм}
%\newtheorem{introduction}[theorem]{Introduction}

\renewcommand{\proof}{\\\textbf{Доказательство.}~}

%\def\startprog{\begin{lstlisting}[language=Java,basicstyle=\normalsize\ttfamily]}

%\theoremstyle{remark}
%\newtheorem{remark}[theorem]{Remark}
%\include{header}
%%%%%%%%%%%%%%%%%%%%%%%%%%%%%%%%%%%%%%%%%%%%%%%%%%%%%%%%%%%%%%%%%%%%%%%%%%%%%%%

\numberwithin{theorem}{chapter}        % Numbers theorems "x.y" where x
                                        % is the section number, y is the
                                        % theorem number

%\renewcommand{\thetheorem}{\arabic{chapter}.\arabic{theorem}}

%\makeatletter                          % This sequence of commands will
%\let\c@equation\c@theorem              % incorporate equation numbering
%\makeatother                           % into the theorem numbering scheme

%\renewcommand{\theenumi}{(\roman{enumi})}

%%%%%%%%%%%%%%%%%%%%%%%%%%%%%%%%%%%%%%%%%%%%%%%%%%%%%%%%%%%%%%%%%%%%%%%%%%%%%%


%%%%%%%%%%%%%%%%%%%%%%%%%%%%%%%%%%%%%%%%%%%%%%%%%%%%%%%%%%%%%%%%%%%%%%%%%%%%%%%

%This command creates a box marked ``To Do'' around text.
%To use type \todo{  insert text here  }.

\newcommand{\todo}[1]{\vspace{5 mm}\par \noindent
\marginpar{\textsc{ToDo}}
\framebox{\begin{minipage}[c]{0.95 \textwidth}
\tt #1 \end{minipage}}\vspace{5 mm}\par}

%%%%%%%%%%%%%%%%%%%%%%%%%%%%%%%%%%%%%%%%%%%%%%%%%%%%%%%%%%%%%%%%%%%%%%%%%%%%%%%

\binoppenalty=10000
\relpenalty=10000

\begin{document}


\bibliographystyle{sty/gost71s}       % Set the bibliography style to AMS
                                % alphabetized. (Can use ``amsalpha'' or
                                % ``abbrv''instead.)

% Begin the front matter as required by Rackham dissertation guidelines

\initializefrontsections

\pagestyle{title}

\begin{center}
Санкт-Петербургский национальный исследовательский университет \\ информационных технологий, механики и оптики

\vspace{2cm}

Кафедра компьютерных технологий

\vspace{3cm}

{\Large А. К. Касс}

\vspace{2cm}

\vbox{\LARGE\bfseries
Массовая задача определения кратчайшего расстояния\\до произвольной
конфигурации отрезков на плоскости}

\vspace{4cm}

Бакалаврская работа 

\vspace{1cm}

{\Large Научный руководитель: А. С. Ковалев}

\vspace{6cm}

Санкт-Петербург\\ 2012
\end{center}

\newpage

\setcounter{page}{2}
\pagestyle{plain}

%\dedicationpage{Put a dedication here}
% Dedication page

%\startacknowledgementspage
% Acknowledgements page
%{Put Acknowledgements here}

% Table of contents, list of figures, etc.
\tableofcontents
%\listoffigures


\def\t#1{\mbox{\texttt{\hbox{#1}}}}
\def\b#1{\textbf{#1}}
\def\tb#1{\t{\b{#1}}}

\def\cln#1{\t{#1}}
\def\pcn#1{\t{#1}}
\newcommand{\p}{\par Здесь будет текст...}

\def\drawfigure#1#2#3{
        \begin{figure}[H]
        \centerline{ \includegraphics{pic/#1}}
        \caption{#2}
        \label{#3}
        \end{figure}
}
\def\drawfigurex#1#2#3#4#5{
        \begin{figure}[#1]
        \centerline{ \includegraphics[#2]{#3}}
        \caption{#4}
        \label{#5}
        \end{figure}
}

% Chapters
\startthechapters
% -*-coding: utf-8-*-
\startprefacepage

Просто крутой текст

\FloatBarrier

%%-*-coding: utf-8-*-
\chapter{Решение задачи поиска кратчайшего расстояния от фиксированной вершины до всех остальных}
\label{chapSVD}

В данной главе описаны алгоритмы по решению классической задаче на графах - поиску кратчайших расстояний от одной вершины до всех остальных. В первой части главы представлен краткий обзор предметной области. Во второй параллельные модификации алгоритма Беллмана-Форда. 

\FloatBarrier
\section{Обзор существующих решений}
\subsection{Алгоритм Дейкстры}

Одним из наиболее заметных алгоритмов для решения данной задачи является алгоритм Дейкстры. Придуманный еще в 1959 году Эдсгером Вибе Дейкстрой он сохраняет свою актуальность и по сей день. Основная идея состоит в последовательном пополнении множества вершин, расстояние для которых уже корректно посчитано. При этом на каждом шаге выбирается вершина, которая находится ближе остальных к уже посчитанному множеству. 

Существует множество модификации алгоритма основанных на различных структурах данных для выбора вершины с минимальный расстоянием на каждом из шагов алгоритма. В зависимости от этого алгоритм может работать O(V*V+E), O(ElogV) или O(VlogV+E). 

Основная проблема алгоритма состоит в том, что он работает только на графах с неотрицательным весом ребер. С этой проблемой справляется алгоритм Беллмана-Форда. 


\FloatBarrier
\subsection{Алгоритм Беллмана-Форда}

Классический алгоритм Беллмана-Форда работает на графах с произвольным весом ребер, однако имеет заметно худшую асимптотику по сравнению с алгоритмом Дейсктры - O(VE). 

Основная идея алгоритма основана на идее динамического программирования. После k итерации алгоритма утверждается, что будут корректно посчитаны и обработаны значения веса путей длиной не более K. И после V итерации расстояние до каждой из вершин посчитано корректно. Ниже приведен каноничный псевдокод алгоритма. 

\FloatBarrier
\begin{algorithm}
\caption{Классический алгоритм Беллмана-Форда}\label{bf_classic_seq}
\begin{algorithmic}[1]
\Procedure{ClassicBellmanFord}{$G,start$}
\State $dist\gets \left\{ {\infty ... \infty}\right\}$
\State $dist[start] \gets 0$
 
\For{$i = 0$ to $|G.vertices| $}
	\For{$e \in G.edges $}
		\State $dist[e.to] \gets \max(dist[e.to], dist[e.from] + e.w)$
	\EndFor
\EndFor
\State \textbf{return} $dist$
\EndProcedure
\end{algorithmic}
\end{algorithm}

\FloatBarrier
Кроме того существует интересная модификация алгоритма, которая поддерживает на каждой итерации набор вершин, расстояние до которых изменились на предыдущем шаге алгоритма. Из очевидных соображений мы имеем право рассматривать только эти и никакие другие вершины. Этот алгоритм на практике зачастую работает заметно быстрее чем классическая версия в некоторых случаях, но об этом подробно будет описано позднее. Будем называть эту версию BFS-подобный Беллман-Форд. Ниже приведен псевдокод алгоритма 

\FloatBarrier
\begin{algorithm}
\caption{BFS-подобный Беллман-Форд}\label{bf_classic_seq}
\begin{algorithmic}[1]
\Procedure{BFSBellmanFord}{$G,start$}
\State $dist\gets \left\{ {\infty ... \infty}\right\}$
\State $dist[start] \gets 0$
\State $VertexSet \gets \left\{ {start}\right\}$\Comment{структура данных для хранения набора вершин} 
\State $NextVertexSet \gets \emptyset$ 
\State {$step \gets 0$ }
\While {$step < |G.vertices|$ \algorithmicand \algorithmicnot $ VertexSet.empty()$}
	\State $step++$
	\State $NextVertexSet.clear()$
	
	\For{$v \in VertexSet$}
		\For{$e \in G.edgesFrom[v] $} \Comment{исходящие ребра из текущей вершины} 
			\If {$dist[e.to] < dist[e.from] + e.w$} 
				\State $dist[e.to] \gets dist[e.from] + e.w$
				\State $NextVertexSet.insert(e.to)$								
			\EndIf
		\EndFor
	\EndFor
	
	\State $VertexSet \gets NextVertexSet$	
\EndWhile

\EndProcedure
\end{algorithmic}
\end{algorithm}

\FloatBarrier
\subsection{Другие алгоритмы}

Также известны специализированные алгоритмы, такие как алгоритм A* и D*, которые оперирует большими специализированными графами и используют ряд эвристик для поиска расстояний. При этом в контексте наших исследований они затронуты не будут.

\FloatBarrier
\section{Параллельный алгоритм Беллмана-Форда}

В предыдущей главе были рассмотрены классические алгоритмы поиска кратчайших путей в графе. В этой главе будет рассмотрен алгоритм Беллмана-Форда в контексте параллельных вычислений. Кроме того будем использовать в каждом из алгоритмов идею ранней остановки - если на текущем шаге ни одно из значений массива расстояний не изменилось, то имеем право выйти из основного цикла. В последующих главах будет представлено несколько версий алгоритма, а также их последующее сравнение и рекомендации по использованию.  

\FloatBarrier
\subsection{Параллелизация по ребрам вершины}
Первая версия алгоритма основана на параллельной обработке всех ребер, исходящих из текущей вершины. Псевдокод, который практически не отличается от классической версии, приведен ниже. 


\FloatBarrier
\begin{algorithm}
\caption{Параллельный Беллман-Форд по ребрам вершины}\label{bf_classic_par1}
\begin{algorithmic}[1]
\Procedure{BellmanFordPar1}{$G,start$}
\State $dist\gets \left\{ {\infty ... \infty}\right\}$
\State $dist[start] \gets 0$
 
\For{$i = 0$ to $|G.vertices| $}
	\State {$changed \gets $ \algorithmicfalse}
	\For{$v \in G.vertices $}
		\algrenewcommand\algorithmicfor{\textbf{parfor}}
		\For{$e \in G.fromEdges[v] $} 
			\If {$dist[e.to] < dist[e.from] + e.w$} 
				\State $dist[e.to] \gets dist[e.from] + e.w$
				\State {$changed \gets $ \algorithmictrue}						
			\EndIf
			\State $dist[e.to] \gets \max(dist[e.to], dist[e.from] + e.w)$
		\EndFor	
		\algrenewcommand\algorithmicfor{\textbf{for}}

	\EndFor
	\If {\algorithmicnot $changed$} 
		\State $break$
	\EndIf

\EndFor
\State \textbf{return} $dist$
\EndProcedure
\end{algorithmic}
\end{algorithm}

\FloatBarrier
\subsection{Параллелизация по всем ребрам}
Идея второго алгоритма состоит в разбиений всего набора вершин на некоторые подмножества, каждое из которых будет обрабатываться отдельным процессором. При этом для каждой вершины будем рассматривать набор ребер, входящих в нее. Это необходимо для того, чтобы обновление фиксированной ячейки в массиве расстояний происходило только одним потоком. 

\FloatBarrier
\begin{algorithm}
\caption{Параллельный Беллман-Форд по всем ребрам}\label{bf_classic_par2}
\begin{algorithmic}[1]
\Procedure{BellmanFordPar2}{$G,start$}
\State $dist\gets \left\{ {\infty ... \infty}\right\}$
\State $dist[start] \gets 0$
\State {$prefsum \gets $ prefix sum of vertices incoming degree} 
\State {$planMap \gets BuildPlan(prefsum, 0, |G.vertices|$ } 

\For{$i = 0$ to $|G.vertices| $}	
	\If {\algorithmicnot $ProcessLayer(G, planMap, prefsum, 0, |G.vertices|)$} 
		\State $break$						
	\EndIf
		
		
\EndFor
\State \textbf{return} $dist$
\EndProcedure

\State 
\Procedure{BuildPlan}{$prefsum, startV, endV$}  \Comment{Функция возвращают структуру, которая по отрезку возвращает его середину по количеству ребер }

\State $edgesNumber \gets prefsum[endV] - prefsum[startV]$
\If {$edgesNumber < threshold$} 
	\State $midV \gets $ binary search on edges number
	\State $resultMap[startV][endV] \gets midV$ 
	\State $resultMap[startV][endV].insert(BuildPlan(prefsum, startV, midV)$ 
	\State $resultMap[startV][endV].insert(BuildPlan(prefsum, midV, endV)$ 					
\EndIf

\State \textbf{return} $resultMap$
\EndProcedure

\State 
\Procedure{ProcessLayer}{$G, planMap, prefsum, startV, endV$}  
\State $edgesNumber \gets prefsum[endV] - prefsum[startV]$
\If {$edgesNumber < threshold$} 
	\State process vertices sequentally 	
\Else	
	\State $midV \gets planMap[startV][endV]$ 
	\State $ProcessLayer(G, planMap, prefsum, startV, midV)$
	\State $ProcessLayer(G, planMap, prefsum, midV, endV)$	
\EndIf

\EndProcedure

\end{algorithmic}
\end{algorithm}

\FloatBarrier
\subsection{Параллелизация BFS - версии}
Предыдущие две версии были основаны на параллелизации классической версии Беллмана-Форда. В основе следующего алгоритма лежит идея обхода в ширину (Алгоритм 2). В качестве основы для параллельной версии такого алгоритма был взят параллельный обход в ширину, предложенный Умутом Акаром и Майком Рэйни. Подробнее о внутреннем устройстве их подхода речь зайдет во 2 части работы в контексте решения задачи поиска расстояний между каждой парой вершин. Пока лишь приведем псевдокод 

\FloatBarrier
\begin{algorithm}
\caption{Параллельный BFS-подобный Беллман-Форд}\label{bf_bfs_par}
\begin{algorithmic}[1]
\Procedure{BellmanFordPar3}{$G,start$}
\State $dist\gets \left\{ {\infty ... \infty}\right\}$
\State $dist[start] \gets 0$
\State $Frontier \gets \left\{ {start}\right\}$\Comment{структура данных для хранения исходящих ребер текущего множества } 

\For{$i = 0$ to $|G.vertices| $}	
	\State $Frontier \gets handleFrontier(Frontier)$ \Comment{релаксируем ребра из фронтира и строим новый} 
	
	\If { $Frontier.empty()$} 
		\State $break$						
	\EndIf
		
		
\EndFor
\State \textbf{return} $dist$
\EndProcedure
\State
\Procedure{HandleFrontier}{$Frontier$}
\State recursively divide current frontier, atomically relax edges in frontier and building a new one

\State \textbf{return} $NewFrontier$
\EndProcedure

\end{algorithmic}
\end{algorithm}


\FloatBarrier
\subsection{Сравнение подходов}
Каждый из вышеизложенных подходов имеет свои особенности, что позволяет каждому из них конкурировать друг с другом на некоторых типах графов. Рассмотрим эти особенности.

С первого взгляда может показаться, что Алгоритм 3 имеет лишь одни недостатки - он имеет наименьшим образом по сравнению с последующими задействует все процессоры и при этом асимптотически равен остальным двум. Однако рассмотрим внимательнее каноничный аглоритм Беллмана-Форда и запустим его на плотном графе, где для каждого ребра верно, что индекс вершины источника меньше индекса вершины назначения. В этом случае каноничной версии достаточно будет сделать лишь две итерации внешнего цикла, поскольку на каждой итераций внутреннего цикла значение расстояния для текущей вершины будет корректно посчитано (очевидно доказывается по математической индукции). Так как количество итерации третьего алгоритма в подобных графах может быть значительным и размер текущей очереди может быть большим, а второму же алгоритму на таких графах неплохая способность параллелиться будет только вредить - она будет заметно увеличивать число итераций внешнего цикла. То есть в таких случаях последние два алгоритма работают хуже первого.   

Но очевидно, что в большинстве случаев последние два алгоритма будут показывать лучшие результаты. Сравним эти два подхода. Алгоритм, основанный на обходе в ширину, заметно сокращает количество вершин для обработки в пределах каждой итерации. Однако на плотных графах количество таких вершин значительно и такой подход показывает себя не с лучшей стороны. 

\FloatBarrier
\subsection{Тестирование}

Для подтверждения вышеприведенных замечаний все вышеизложенные подходы были реализованы на основе библиотеки для параллельных вычислений PASL. Тестирование производилось на ряде графов на машине 40-core Intel machine (with hyper-threading) with 4×2.4GHz Intel
10-core E7-8870 Xeon processors, a 1066MHz bus, and 256GB of
main memory. 

ИСПРАВИТЬ ДАННЫЕ В ТАБЛИЦАХ (раскидать RandomDense и RandomSparse)
\begin{table}
\centering

\begin{tabular}{l|ccc|cc|cc|ccc|ccc}  
Algo №& \multicolumn{3}{c}{Complete} & \multicolumn{2}{c}{BalancedTree} & \multicolumn{2}{c}{SquareGrid} & \multicolumn{3}{c}{RandomSparse} & \multicolumn{3}{c}{RandomDense}\\
& TS & + & - & 0.5 & 1 & + & +- & 0.5+ & 0.5K+ & 0.5- & 0.5- & 0.96+ & 0.96+\\
\hline\hline
3 & 2.43 & 4.65 & nc & 116.31 & 9.04 & 5.49 & 13.40 & nc & nc & nc & nc & 24.35 & 5.01 \\  
4 & 5.17 & 0.18 & 10.84 & 3.59 & 3.08 & 5.92 & 7.10 & 2.77 & 0.48 & 14.68 & 6.38 & 2.42 & 0.46 \\
5 & 44.63 & 0.37 & 23.55 & 0.44 & 0.31 & 4.42 & 0.58 & 0.98 & 0.60 & 22.59 & 10.25 & 0.76 & 0.71 \\
\hline
\end{tabular}

\caption{Bellman-Ford algorithms comparison}
\label{graph_description}
\end{table}

\begin{table}
\centering

\begin{tabular}{c|c|c|c}  
Name & Vertices & Edges & Description\\
\hline\hline
Complete TS & 1 & 1 & 1 \\  
Complete sign & 1 & 1 & 1 \\  
BalancedTree fraction & 1 & 1 & 1 \\  
SquareGrid sign & 1 & 1 & 1 \\  
RandomSparse fraction sign & 1 & 1 & 1 \\  
RandomDense fraction sign & 1 & 1 & 1 \\  

\hline
\end{tabular}

\caption{Input graph description}
\label{bf_algo_comparison}
\end{table}



\FloatBarrier
\section{Выводы}

TODO

\FloatBarrier

%%-*-coding: utf-8-*-
\chapter{Решение задачи поиска расстояний между каждой парой вершин графа}

В этой главе будет приведено решение задачи поиска расстояний между каждой парой вершин. В начале главы будет краткий обзор предметной области, после будут приведены два наивных решения для поиска пути, а после комбинацией будет приведено решение задачи для социальных неориентированных невзвешенных графов, которое будет сочетать несколько подходов и идей, изложенных в предыдущих алгоритмах. 

\FloatBarrier
\section{Обзор существующих решений}

\subsection{Алгоритм Флойда}
Одним из наиболее известных алгоритмов, который применяется для решения данной задачи является алгоритм Флойда. Этот алгоритм использует идею динамического программирования и выполняется за O(VVV). Основная идея состоит в обновлений пути между двумя текущими вершинами выбором некоторой вершины, через который может пройти потенциальный кратчайший путь. Псевдокод алгоритма приведен ниже. 

\FloatBarrier
\begin{algorithm}
\caption{Алгоритм Флойда}\label{floyd}
\begin{algorithmic}[1]
\Procedure{Floyd}{$G$}
\State $dist\gets \left\{ {   \left\{ {\infty ... \infty}\right\}  ... \left\{ {\infty ... \infty}\right\} }\right\}$
\For{$e \in G.edges $}
	\State $dist[e.from][e.to] \gets e.w$
\EndFor 
\State
\For{$i = 0$ to $|G.vertices| $}
	\For{$u = 0$ to $|G.vertices| $}
		\For{$v = 0$ to $|G.vertices| $}
			\State $dist[u][v] \gets \min(dist[u][v], dist[u][i] + dist[i][v])$
		\EndFor
	\EndFor
\EndFor
\State \textbf{return} $dist$
\EndProcedure
\end{algorithmic}
\end{algorithm}

\FloatBarrier
\subsection{Альтернативы}
В некоторых случаях оказываются эффективны другие подходы. Например, можно для каждой вершины по отдельности запустить некоторых алгоритм поиска кратчайшего пути до всех остальных вершин. Для случая неотрицательных ребер можно применить алгоритм Дейкстры, в более общем случае может быть применен Беллман-Форд. Кроме вышеприведенных подходов также известен алгоритм Джонсона, который работает на графах без циклов отрицательного веса и находит кратчайшие расстояния за время O(VV * log(V) + VE). Все эти подходы оказываются эффективны в случае разреженных графов.

В последующих подходах в качестве основы для параллельного алгоритма будет использоваться именно идея подсчета расстояний либо для каждой вершины по отдельности, либо подсчета расстояний для групп вершин. И все нижеперечисленные алгоритмы, как и описанные выше альтернативы хорошо работает на разреженных графах.

\FloatBarrier
\section{Наивная параллельная версия}
Первая версия заключается исключительно в запуске Беллмана-Форда для каждой из вершин. При этом заметим, что так как каждый из них независим друг от друга, то можем эти запуски распараллелить между собой. Таким образом, псевдокод из себя представляет следующее


\FloatBarrier
\begin{algorithm}
\caption{Наивная параллельная версия}\label{all_pairs_par1}
\begin{algorithmic}[1]
\Procedure{AllPairsPar1}{$G$}
\State \textbf{return} $HandleVertices(G, 0, |G.vertices|)$
\EndProcedure
\State
\Procedure{HandleVertices}{$G, startVertex, endVertex$}

\If {$endVertex - startVertex < threshold$} 
	\State run parallel Bellman-Ford for every vertex 
	\State \textbf{return} $distances$	
\Else	
	\State $midV \gets (startVertex + endVertex) / 2$ 
	\State $HandleVertices(G, startV, midV)$
	\State $HandleVertices(G, midV, endV)$	
\EndIf

\EndProcedure

\end{algorithmic}
\end{algorithm}

\FloatBarrier
\section{Параллельный алгоритм для объединенного графа}
Развитием предыдущей идеи является наблюдение, что для некоторого набора вершин можем построить общий граф и запустить на нем Беллмана-Форда, что потенциально может повысить производительность за счет высокой параллельности каждого отдельно взятого Беллмана-Форда. Кроме того, это избавит нас от выбора константы для предыдущей версии, что упростит использование алгоритма для пользователя. 

Идея заключается в запуске алгоритма Беллман-Форд на графе, вершины которого описываются двумя значениями - текущей вершины в обходе и вершины, из которой этот обход начался (иными словами, вершины, из которой мы ищем кратчайшие расстояния). После построения графа будет достаточно запустить обход, при этом положив в Frontier все вершины вида (i, i). В итоге кратчайшее расстояние для вершины (i, j) будет интерпретироваться как кратчайшее расстояние от вершины i до вершины j в исходном графе. 

Однако, как будет показано позднее, такой подход на практике оказался медленнее наивной версий. Но при этом идея обработки ряда вершин одновременна легла в основе следующего алгоритма для социальных графов. 
\FloatBarrier
\section{Параллельный алгоритм для социальных графов}
В данной главе будет рассмотрен алгоритм поиска кратчайшего пути между каждой парой вершин для графов реальных социальных сетей. При этом рассмотренный граф будет невзвешенный и неориентированный. Кроме того, в этой главе будет затронута структура данных, уже ранее используемая в параллельном обходе в ширину. Будет описан ее интерфейс и принцип работы. 

\FloatBarrier
\subsection{Идея алгоритма}
В графах для социальных сетей известна одна эвристика, которая называется "Теория шести рукопожатий". В ее основе лежит тот факт, что практически любые два человека на земле знакомы не более, чем через пятерых промежуточных людей. Таким образом, выбрав некоторую случайную вершину, мы сможем добраться от нее до большинства других вершин не более, чем за 6 ребер. Воспользуемся этой эвристикой в нашем алгоритме и выберем вершину наибольшей степени в качестве базовой. И будем обрабатывать два множества различным образом - для меньшего (вершины, которые находятся на расстоянии больше шести) будем запускать параллельный обход в ширину для каждой вершины, для большого - воспользуемся методом динамического программирования для подсчета ответа. 

Но прежде, чем приступить к описанию непосредственно алгоритма опишем структуру данных, предложенную Майком Рэйни и Умутом Акаром для обработки множества ребер на нескольких ядрах - Фронтир (Frontier, англ.)

\FloatBarrier
\subsection{Структура данных Фронтир}
Структура данных подробно рассмотрена в статье Умут Акара. Здесь же приведено краткое ее описание, основные принципы работы и интерфейс. Фронтир представляет из себя некоторый набор ребер. При этом он поддерживает операций разделения множества пополам, слияния множеств, добавления ребер вершины и итерирования по ребрам. При этом  операций слияния и разбиения выполняются за время пропорциональное O(log n), добавление ребер вершины происходит за константу, а итерирование за константу для каждого ребра. Такая асимптотика достигается за счет лежащей в основе bootstrapped chunked sequence, которая представляет из себя последовательность, где каждому элементу сопоставляется его вес. И операций слияния и разбиения выполняются в соответсвии с этими весами и выполняются за O(log n). Более подробное описание bootstrapped chunked sequence приведено в статье. 

\FloatBarrier
\subsection{Работа алгоритма}
Как уже было отмечено ранее, работа алгоритма разбивается на три этапа - анализ графа и выбор базовой вершины, обработка меньшего множества и обработка большего. Разберемся с каждым этапом по отдельности.

Первый и самый простой этап состоит в выборе базовой вершины. В качестве нее будет выбрана вершина с наибольшей степенью. После этого из этой вершины будет запущен обход в ширину, который найдет все вершины, отстоящие не более, чем на K (в описанном в предыдущем пункте случае K = 6). Таким образом, все такие вершины попадают в множество, которое будет обработано на третьем этапе алгоритма. Все остальные вершины попадают в второе множество. Псевдокод этого этапа выглядит следующим образом

\FloatBarrier
\begin{algorithm}
\caption{Первая фаза алгоритма}\label{all_pairs_social1}
\begin{algorithmic}[1]
\Procedure{ConstructSets}{$G$}
\State {$baseVertex \gets $ vertex with max degree}
\State {$dist \gets $ run serial bfs from $baseVertex$}
\State $handleByBaseVertexSet \gets \emptyset$
\algrenewcommand\algorithmicfor{\textbf{parfor}} 
\For{$i = 0$ to $|G.vertices| - 1$}
\algrenewcommand\algorithmicfor{\textbf{for}}
	\If {$dist[i] \leq K$ }
		\State $handleByBaseVertexSet.add(i)$
	\EndIf
\EndFor

\State $otherVertexSet \gets G.vertices \setminus handleByBaseVertexSet$ 
\State \textbf{return} $ baseVertex, handleByBaseVertexSet, otherVertexSet$
\EndProcedure

\end{algorithmic}
\end{algorithm}




В качестве алгоритма для обработки второго множества запустим параллельный обход в ширину (он же Беллман-Форд) для каждой из вершин этого множества. Кроме того запустим обход для базовой вершины. Эти значения расстоянии нам помогут на третьем этапе алгоритма. На этом этапе используется просто Алгоритм 7 для множества otherVertexSet. 

Для поиска искомых значении для вершин множества handleByBaseVertexSet воспользуемся методом динамического программирования. Но сперва обсудим основные принципы построения алгоритма и доказательство его корректности. 

Рассмотрим некоторую вершину, которая находится на расстояний D от базовой вершины (D <= K). То какие вершины для нее могут находится на расстоянии I? Это могут быть только те вершины, которые находятся на расстоянии [I-D, I+D] от базовой. Иначе бы не выполнялось свойство, что путь кратчайший. С другой стороны, если рассмотреть некоторую вершины, расстояние до которой равняется I, то для всех вершин из множества handleByBaseVertexSet верно, что кратчайшее расстояние от них до нее варьируется в промежутке [I-K, I+K]. 

 Предположим, что мы запустили обход в ширину из всех вершин большого множества. То какие вершины могут быть в слое с номером I? Ответ вытекает из рассуждений предыдущего абзаца - только вершины, расстояние от которых до базовой варьируется в промежутке [I-K, I+K]. То есть каждая из вершин будет принимать участие в не более, чем 2K+1 слоях. То построим для каждого слоя обхода в ширину множество возможных вершин на этом слое. Это избавит нас от построения следующего фронтира по текущему. И при этом общее количество вершин во всех слоях будет пропорционально числу вершин в графе (если учитывать, что K - небольшое число, меньшее 7). 
 
 После того, как мы построили набор вершин для каждого из слоев мы можем воспользоваться структурой Фронтир для эффективного распараллеливания процесса обработки ребер, исходящих из этих вершин. Но к текущему моменту мы никак не воспользовались тем фактом, что вершины расположены близко друг к другу, и, может быть, существует способ для оптимальной обработки группы вершин. Такой подход существует и основан на идее динамического программирования и применения битовых векторов. 
 
 Будем поддерживать две динамики. Значениями в полях массива будут битовые вектора - это некоторая структура, где каждый из битов соответсвует вершине из множества handleByBaseVertexSet. Первая из динамик mask[u][i] - это битовый вектор вершин, расстояние от которых до U равно I. Вторая из динамик calc[u][i] - набор вершин, расстояние от которых до U не более, чем I. 
 
 Рассмотрим процесс пересчета значений динамики. Для подсчета текущего значения mask воспользуемся формулой (2.1). Обратим внимание, что в битовые вектора должны поддерживать битовые логические операции. 
  
\begin{equation}
mask[v][i] = (OR (mask[u][i - 1] where \exists edge u->v)) AND (NOT calc[v][i - 1])
\end{equation}

В свою очередь calc пересчитывается согласно (2.2)
\begin{equation}
calc[v][i] = calc[v][i - 1] OR mask[v][i]
\end{equation}

Псевдокод пересчета значений динамики представлен ниже. 

\FloatBarrier
\begin{algorithm}
\caption{Пересчет динамики}\label{all_pairs_social2}
\begin{algorithmic}[1]
\State $ K \gets 6$
\State

\Procedure{CalculateDistancesForBigSet}{$G, baseVertex, handleByBaseVertexSet$}
\State {$maxLayer \gets $ calculate max distance from baseVertex}
\State {$Frontier[maxLayer + K] \gets \left\{ {Frontier() ... Frontier()}\right\} $} \Comment{Фронтир для каждого уровня обхода в ширину. Изначально пустой} 
\State {$VertexSet[maxLayer + K] \gets \left\{ {VertexSet() ... VertexSet()}\right\} $} \Comment{Набор вершин для каждого уровня} 

\For{$i = 0$ to $|G.vertices| $}
	\For{$j = dist[baseVertex][i] - K$ to $dist[baseVertex][i] + K$}		
		\State $Frontier[j].pushEdgesOf(i)$
		\State $VertexSet[j].addVertex(i)$
	\EndFor
\EndFor

\For{$i = 0$ to $maxLayer + K$}	
	\For{$j = dist[baseVertex][i] - K$ to $dist[baseVertex][i] + K$}		
		\State $Frontier[j].pushEdgesOf(i)$
	\EndFor
\EndFor
\State
\State $mask\gets \left\{ {   \left\{ {bitVector(0) ... bitVector(0)}\right\}  ... \left\{ {bitVector(0) ... bitVector(0)}\right\} }\right\}$ \Comment{Обратим внимание, что для каждой вершины нам нужно хранить всего 2 * K + 1 битовых векторов. Однако для упрощения понимания псевдокода будем обращаться к I слою вершины U просто как mask[u][i]} 
\State $calc\gets \left\{ {   \left\{ {bitVector(0) ... bitVector(0)}\right\}  ... \left\{ {bitVector(0) ... bitVector(0)}\right\} }\right\}$

\For{$v \in handleByBaseVertexSet $}
	\State $mask[v][0] \gets bitVector(id(v))$ \Comment{Будем считать, что существует функция id, которая по номеру вершины возвращает соответсвующий ей бит} 

\EndFor 
\State
\For{$layerToCalc = 1$ to $maxLayer + K$}
	\State $frontierLayer \gets layerToCalc - 1$ \Comment{Для пересчета значений I слоя нам необходимо использовать вершины с предыдущего слоя} 
	\State $ProcessLayerLazy(G, Frontier[frontierLayer], mask, layerToCalc)$ 

	\algrenewcommand\algorithmicfor{\textbf{parfor}}
	\For{$v \in VertexSet[layerToCalc] $}
		\State $calc[v][layerToCalc] \gets mask[v][layerToCalc])$ 
		\If {not first layer for vertex $v$}
			\State $calc[v][layerToCalc - 1] \gets \lnot calc[v][layerToCalc - 1]$ 
			\State $mask[v][layerToCalc] \gets mask[v][layerToCalc] \land calc[v][layerToCalc - 1]$ 
			\State $calc[v][layerToCalc - 1] \gets \lnot calc[v][layerToCalc - 1]$ 
			\State $calc[v][layerToCalc] \gets calc[v][layerToCalc] \lor calc[v][layerToCalc - 1]$ 
		\EndIf
	\EndFor 
	\algrenewcommand\algorithmicfor{\textbf{for}}	
\EndFor 
\EndProcedure
\end{algorithmic}
\end{algorithm}
\FloatBarrier

Псевдокод обработки текущего Фронтира представлен ниже. 
\begin{algorithm}
\caption{Обработка Фронтира}\label{all_pairs_social3}
\begin{algorithmic}[1]
\State $ K \gets 6$
\State
\Procedure{ProcessLayerLazy}{$G, Frontier, mask, layer$}
\State blablalba
\EndProcedure
\end{algorithmic}
\end{algorithm}


\FloatBarrier
Наконец, как по имеющимся данным динамики восстановить ответ? Для каждого значения mask[u][i] найдем единичные биты в маске. Установленный в единицу бит J говорит о том, что расстояние от вершины из "большого" множества с идентификатором J до вершины u равно i. Таким образом, мы сможем полностью восстановить ответ для каждой вершины. 


Итого, псевдокод алгоритма выглядит следующим образом.

\FloatBarrier
\begin{algorithm}
\caption{Параллельная версия для социальных графов}\label{all_pairs_social}
\begin{algorithmic}[1]
\State $ K \gets 6$
\State

\Procedure{AllPairsSocialPar}{$G$}
\State $dist\gets \left\{ {   \left\{ {\infty ... \infty}\right\}  ... \left\{ {\infty ... \infty}\right\} }\right\}$
\State {$ baseVertex, handleByBaseVertexSet, otherVertexSet \gets ConstructSets(G)$}
\State {$ AllPairsPar1(G, otherVertexSet, dist)$}
\State {$ CalculateDistancesForBigSet(G, baseVertex, otherVertexSet, dist)$}


\State Заполнение массива расстояний
\algrenewcommand\algorithmicfor{\textbf{parfor}}
\For{$v = 0$ to $|G.vertices|$}
\algrenewcommand\algorithmicfor{\textbf{for}}
	\For{$j = dist[baseVertex][i] - K$ to $dist[baseVertex][i] + K$}		
		\State $Frontier[j].pushEdgesOf(i)$
	\EndFor
\EndFor 
\State \textbf{return} $dist$ 
\EndProcedure

\end{algorithmic}
\end{algorithm}


\FloatBarrier
\subsection{Сравнение с наивными версиями}
бла-бла-бла.

\FloatBarrier
\section{Выводы}
Всякие разные выводы бла-бла-бла.

\FloatBarrier

%%-*-coding: utf-8-*-
\chapter{Практическая реализация алгоритма}

В рамках данной работы была написана утилита для поиска корректно-синхронизированных полей, использующая алгоритм из предыдущего раздела. Программа написана на языке программирования java и интегрирована с jDRD.

\FloatBarrier
\section{Особенности реализации}
Данная программа имеет два режима работы: \emph{локальный} и \emph{локальный}. \emph{Локальный} режим подразумевает, что анализируемый код
может использоваться строронними приложениями, не входящими в область анализа. Таким образом, если поле корректно-синхронизированно в рамках анализируемой программы, но оно доступно для изменения, то в локальном режиме оно не считается корректно-сихронизированным. Такой режим нужен для анализа различных библиотек. \emph{Глобальный} режим подразумевает, что 
анализируемый код, никем используется. Если поле корректно-синхронизированно в рамках анализируемой программы, то даже если оно доступно для изменения, глобальный режим отметит отметит его как корректно-синхронизированное. Данный режим необходим для тестирования законченых приложений. Множество переменных, выделенных в глобальном режиме включает в себя множество, выделенное при работе в локальном режиме.



В листинге \ref{Account} поле $balance$, корректно-синхронизирован и будет выделено при работе программы и в локальном и в глобальном режиме. Если убрать модификатор $private$ у поля $balance$ и предположить, что в рамках программы снаружи к полю
$balance$ не обращаются, то оно будет корректно-синхронизировано только с точки зрения глобального режима.

\lstinputlisting[caption=Пример корректно-синхронизированного поля для локального режима, language=Java, label=Account]{code/Account.java}

Программа на вход принимает скомпилированные фалйы java программ. На вход ей можно подать либо $jar$-файл с программой либо указать путь до папки с $class$-файлами.  Промежутончое представление и граф потока исполнения получено на основе байт-кода с помощью библиотеки $Soot$. Промежуточное представление используемое в данной работе для анализа называется $Shimple$.


\FloatBarrier
\section{Тестирование}
Было проведено тестирование данной программы на различных тестах. Создан набор тестов, покрывающий большинство конструкций java программ. В качестве тестов были использованы программы использующие различные операции синхронизации, обработку ошибок(exception), статические и нестатические блокировки и поля, внутренние классы и т.д.

Также были проведены запуски на реальных библиотеках и приложениях с последующей проверкой результатов.

\FloatBarrier
\section{Сбор статистики и интеграция с jDRD}
В программу статического анализа был добавлен модуль сбора статистики, который подсчитывает различные метрики работы алгоритма. Этот модуль необходим для оценивания результата работы программы.

jDRD получает список корректно-синхронизированных полей через конфигурационный файл, генерируемый разработанной программой.
На стороне jDRD был также включен сбор статистики, который отслеживает количество обращений к корректно-синхронизированным полям, выделенным статическим анализом.


\section{Полученные результаты}
Программа была запущена на различных приложениях, которые активно используют конкурентный доступ к данным .  Результаты приведены в таблице. 

\begin{table}[H]
\label{results}
\begin{center}
\begin{tabular}{|c|c|c|c|}
\hline
Приложение & Общее количество полей & Корректно-синхронизированных полей & Процент \\
\hline
dxFeed & 5730 & 493 & 8,6 \\
\hline
MARS & 4437 & 340 & 7,6 \\
\hline
Tomcat & 7600 & 390 & 5,1 \\

\hline

\end{tabular}
\captionsetup{justification=centering}
\caption{Полученные результаты}
\end{center}
\end{table}
Далее представлено краткое описание тестируемых библиотек.
\\MARS(Monitoring and reporting system) --- Система мониторинга реального времени. Используется для отображения различных данных приложения. 
\\ dxFeed --- система, отвечающая за быструю доставку больших данных(котировок).
\\ Tomcat --- контейнер сервлетов. Позволяет запускать веб-приложения.

По результатам видно, что для данных библиотек  в среднем статическим анализом можно обнаружить около

Далее приведены найденные часто используемые паттерны, применяемые для защиты блокировкой операций блокировкой.
\\
\\
\\
\\
\lstinputlisting[caption=Использование syncronized методов, language=Java, label=Pattern1]{code/Pattern1.java}
\lstinputlisting[caption=Использование блокировки для синхронизации, language=Java, label=Pattern2]{code/Pattern2.java}
\lstinputlisting[caption=Использование блокировки для синхронизации, language=Java, label=Pattern3]{code/Pattern3.java}

\FloatBarrier

%\chapter{Применение}
\section{Решение задачи обратного геокодирования}
Задача обратного геокодирования (reverse geocoding) состоит в
определении адреса по координатам точки.

Входными данными к этой задаче обычно является множество объектов
с известными географическими адресами. Поиск страны и некоторых других частей
адреса осуществляется за счет локализации в подразбиении карты мира на страны, 
страны на регионы и так далее. Для решения данной задачи известно множество методов.
Локализация на уровне улиц происходит иным образом. Определение улицы, рядом с 
которой расположена точка, осуществляется за счет поиска ближайшей улицы или дома (рис. \ref{rgeocoding}).

Эти данные представимы в виде отрезков, поэтому можно предобработать данные карты,
поместив геометрию карты в структуру для быстрого поиска ближайших отрезков, и запомнив
соответствие между геометрическими данными и их метаинформацией.

В результате, с помощью реализованного алгоритма, можно быстро решать задачу обратного геокодирования.

\drawfigure{rgeocoding}{Ближайший объект}{rgeocoding}

\FloatBarrier
\section{Решение задач интерполяции}
Быстрый поиск ближайших отрезков часто используется при интерполяции
функций, заданных на линейных объектах. Например, можно
восстанавливать высотную модель по изолиниям.

Также, зная расстояние до ближайшей линии, можно варьировать параметры
интерполяции. В работе \cite{NGRID} описано применение кампанией Транзас алгоритма
поиска ближайшего отрезка для восстановления рельефа по
картографическим данным. Они накладывают на известные данные шумы,
зависящие от расстояния до изолиний. Вблизи изолиний используется
высокочастотный шум низкой амплитуды, симулирующий мелкие неровности
рельефа, тогда как на удалении от них используется низкочастотный шум
большей амплитуды, позволяющий получать холмистую местность. Этот
подход позволяет по картографическим данным получать реалистичный
рельеф.

В этой же работе описано использование информации о ближайших отрезках
для простого текстурирования рельефа. В зависимости от расстояния до
линейных объектов применяются различные текстуры, которые плавно
переходят друг в друга. Например, в зависимости от расстояния до
береговой линии, сначала может идти текстура песка, а потом травяной растительности.

\FloatBarrier

%\startconclusionpage
В ходе данной работы был получен алгоритм для поиска корректно-синхронизированных полей.
Также была разработана программа реализующая данный алгоритм. 
Программа была запущена на различных библиотеках и показала следующие результаты  : .
Текст разный \cite{DRD}. 

\FloatBarrier


%\startappendices
%\label{appendix}
%\input{appendix}

\bibliography{thesis}

\end{document}
